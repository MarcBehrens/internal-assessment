\section{Tasks and Recommendations}

\subsection{Tasks for the openETCS Work Packages}
\bigskip

\begin{description}
\item[T1]	WP members shall go through all open points in this assessment report.
\item[T2]	All required Documentation – as in 3.2a defined – need to be reviewed or produced.
\item[T3]	The SW development process and its artifacts shall be checked against all A-Tables of the CENELEC Standard for conformity.
\end{description}

\subsection{Recommendation}

Due to the fact that software in the railway signaling sector is mainly developed and tested in traditional ways, the assessors encourage the goals of the openETCS project in terms of ap-plying the formal methods approach for further coming ETCS projects so that certification bod-ies get more familiar with FM development methodology.

Deviating from the primary goal of applying the Open Proofs concept for developing a tool chain, which should have led to the result of the openETCS project (Costs reduction, formal-ized ETCS requirements, transparency of a complex safety-related ETCS, …) the assessors strongly support the decision of using the closed source development tool SCADE Suite and SCADE System for further follow-up projects.

Also for follow-up projects in the rail sector the assessors recommend a combination of con-ventional and agile development processes. For instance, all planning phases can be adapted in consecutive phases like in the V-Model. The actual software development should then be accomplished in an agile development process. (See also Figure 5) \FIXME{figurenr}

The assessors recommend a walk-through through all the tasks as stated in the section 7.1 and solve the open points.