\section{Summary}

\bigskip
The assessors were charged by the openETCS project management board to assess the process of the software development of the openETCS according to the EN 50128:2011.

The main outcomes of the project are the openETCS software and the openETCS tool chain. A high degree of standardization of a vital part of the ETCS System Requirement Specifica-tion prose (normal textual language) was reached by applying the formal methods (FM) con-cept. This has been resulting in a functional interpretation of the ETCS software. For a SIL level 4 software development the EN 50128:2011 has classified the Formal Methods as “Highly Recommended” as for the techniques/measures to
\begin{itemize}
\item	software requirements specification,
\item	software architecture,
\item	software design and implementation,
\item	verification and testing.
\end{itemize}

Since the openETCS toolchain software lacks the maturity for industrial use and the con-straints the standard EN 50128:2011 entails, the PMB has decided to pursue the openETCS SW development with the closed source Tool SCADE Systems and SCADE Suite. This has led to major progress towards developing the ETCS software with the FM approach.

Still the developed software does not cover all the ETCS requirement specifications. The quality of the software development process is not sufficient for the required safety integrity level. A lot of work still needs to be done to fulfill the CENELEC standard to a 100\% in the area of the development process.
